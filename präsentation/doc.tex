\documentclass{beamer}

\usepackage[T1]{fontenc}
\usepackage[ngerman]{babel}
\usepackage[utf8]{inputenc}
% diese drei Pakete
% in dieser
% Reihenfolge
\usepackage{tabularx}
\usepackage{csquotes}
\beamertemplatenavigationsymbolsempty

\author{Hier eure Namen eintragen, Philipp}
\title{Clickbait Detection}
\begin{document}
%\item[\textbf{+}] #1
\newcommand{\pro}[1]{\item[\colorbox{green}{\textcolor{white}{\makebox(5.5,7){\textbf{+}}}}] #1}
\newcommand{\con}[1]{\item[\colorbox{red}{\textcolor{white}{\makebox(5.5,7){\textbf{--}}}}] #1}
\begin{frame}
\titlepage
\end{frame}

\section{Albacore Clickbait Detection}
\begin{frame}
\frametitle{Albacore Clickbait Detection}
    \begin{definition}
	Eine Vorschau eines Nachrichtenartikels, die einen irreführenden Titel oder übertriebenen Inhalt enthält, um falsche Erwartungen 
	beim Benutzer zu erzeugen, heißt Clickbait.
	\end{definition}
\end{frame}
\begin{frame}
    \frametitle{Zielsetzung}
	\begin{itemize}    
	\item Regressor Modell vorstellen, das die Wahrscheinlichkeit angibt, mit der ein Post Clickbait ist.
	\item Mean Sqaured Error auf Datensatz der Clickbait Challenge minimieren.
	\end{itemize}
\end{frame}
\begin{frame}
    \frametitle{Beschreibung}
    \begin{figure}
  	\includegraphics[width=0.8\linewidth]{albacore.png}
\end{figure}
\end{frame}
\begin{frame}
    \frametitle{Evaluierung}
    \center
    \begin{tabular}{c|c}
    Metrik & Ergebnis \\ \hline 
    Mean Squared Error & 0.0315 \\
	Median Absolute Error & 0.122 \\
	F1 Score & 0.670 \\
	Precision & 0.732 \\
	Recall & 0.619 \\
	Accuracy & 0.855 \\
	R2 Score & 0.571 \\
	Runtime & 00:01:10 \\
    \end{tabular}
	
	\begin{itemize}
	\item Clickbait Challenge 2017 gewonnen
	\end{itemize}
\end{frame}
\begin{frame}
    \frametitle{Kritik}
    \begin{itemize}
	\pro{Kleiner Mean Squared Error}
	\pro{Einfaches Modell}
	\con{Nur \texttt{postText} benutzt}
	\end{itemize}
\end{frame}

\section{Zingel Clickbait Detector}
\begin{frame}
\frametitle{Zingel Clickbait Detection}
    \begin{definition}
    %Headers of news articles that have misleading titles and exaggerate the content of the news
	%articles to create misleading expectations for users are called clickbaits   
	Clickbait bezeichnet etwas (wie z.B. eine Überschrift), das so entworfen wurde, dass Leser auf einen Hyperlink klicken wollen,
	insbesondere, wenn der Link zu Inhalten dubiosen Interesses oder Wertes führt.
	\end{definition}
\end{frame}
\begin{frame}
    \frametitle{Zielsetzung}
	\begin{itemize}    
	\item Automatisierten Ansatz entwickeln, um Clickbait aus Twitter Stream zu filtern.
	\item Mean Sqaured Error auf Datensatz der Clickbait Challenge minimieren.
	\end{itemize}
\end{frame}
\begin{frame}
    \frametitle{Beschreibung}
    \begin{figure}
  	\includegraphics[width=0.7\linewidth]{zingel.png}
\end{figure}
\end{frame}
\begin{frame}
    \frametitle{Evaluierung}
    \center
    \begin{tabular}{c|c}
    Metrik & Ergebnis \\ \hline 
    Mean Squared Error & 0.0333 \\
	F1 Score & 0.683 \\
	Accuracy & 0.856 \\
	Runtime & 00:03:27 \\
    \end{tabular}
	
	\begin{itemize}
	\item Zweiter Platz bei Clickbait Challenge 2017
	\end{itemize}
\end{frame}
\begin{frame}
    \frametitle{Kritik}
    \begin{itemize}
	\pro{Kleiner Mean Squared Error}
	\pro{Einfaches Modell}
	\con{Komplizierter als Albacore, trotzdem schlechter}
	\con{Nur \texttt{postText} benutzt}
	\end{itemize}
\end{frame}

\section{Identifying Clickbait Posts on Social Media with an Ensemble of Linear Models}
\begin{frame}
	\frametitle{Identifying Clickbait Posts on Social Media with an Ensemble of Linear Models}
	\begin{definition}
	Definition Clickbait, Erklärung CB Challenge
	\end{definition}
\end{frame}
\begin{frame}
	\frametitle{Approach}
        \begin{itemize}
		\item Zusätzliche Daten gesammelt:
		    \begin{itemize}
			\item Clickbait: FB-Gruppen von Buzzfeed, Upworthy, etc.
			\item Non-Clickbait: FB Gruppen von CNN News, Wikinews, NYTimes.
			\end{itemize}
    	\end{itemize}
\end{frame}
\begin{frame}
	\frametitle{Approach cont.}
        \begin{itemize}
		\item Modeling:
		    \begin{itemize}
    		\item Repeated sentences entfernt
			\item HTML Tags entfernt
			\item Englische Stop-Words entfernt
			\item Stem der Wörter extrahiert
			\item Numbers by token [n] ersetzt
			\item Bag-of-Words um textualle Information in Vector Space zu kodieren. Tokens, 2- und 3-grams benutzt
			\item Keine Image-Daten
			\end{itemize}
		\end{itemize}
\end{frame}
\begin{frame}
	\frametitle{Approach cont.}
        \begin{itemize}
		\item ML Models:
		    \begin{itemize}
		    \item Mehrere LMs kombiniert mit Tree-based-Model.
			\item Für jedes Text-Feature (posttext, targetkeywords, target caption etc.) wurde ein SVM trainiert. Sollten \enquote{clickbaitness} schätzen. Bestes Modell war basiert auf Posttext.
		    \item Weiteres Modellset trainiert, um Standard Deviation der Clickbaitness zu predicten. 
			\item Zuletzt Binärer Classifier gebaut (ebenfalls SVM).
			\item Nach Training der Modele anhand "stacking" kombiniert - Output der Modelle wird als Input für anderes Modell genutzt. Extremely Randomized Trees genutzt.
			\end{itemize}
        \end{itemize}
\end{frame}

\end{document}
